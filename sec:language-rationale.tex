\section{The rationale behind the language}

The choice of thinking of a graph as a triple $G = (V, E, \psi)$ has its
advantages and disadvantages. If we were only concerned with simple graphs,
we could have simplified our notation somewhat by omitting the function
$\psi$, and letting $E$ itself be a subset of the set of 
unordered pairs of distinct elements of $V$. In the case of general graphs,
however, where there can be multiple edges between two vertices, this is
somewhat less convenient. We could persist with this approach by saying
that $E$ be a multisubset instead of a subset, however, this is a little
bit less convenient later when we wish to talk about colorings or
labellings of edges.

An alternate way of defining things could be as follows: Instead of
defining the function $\psi$ as the fundamental concept, one may instead
define the notion of \textbf{incidence} as the fundamental concept as
follows:

\begin{defn} \label{griph}
A griph $G$ is an ordered triple $(V, E, \alpha)$ consisting of a set of
vertices $V$, a set of edges $E$, and a set of ordered pairs $\alpha
\subset V \times E$ such that \item for every $e \in E$, there is at least
one, and at most two elements $v \in V$ such that $(v, e) \in \alpha$. If
$(v, e) \in \alpha$, we say that $v$ is incident to $e$. A griph is called
simple if every edge is incident to exactly two vertices.
\end{defn}


