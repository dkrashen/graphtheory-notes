
\documentclass[12pt]{report}

%%%%%%%%%%%%%%%%%%%%%%%%%%%%%%%%%%%%%%%%%%%%%%%%%%%%%%%%%%%%%%%%%%%%
% package and document formatting stuff
%%%%%%%%%%%%%%%%%%%%%%%%%%%%%%%%%%%%%%%%%%%%%%%%%%%%%%%%%%%%%%%%%%%%

\usepackage{amsmath,amsthm,amsfonts,amscd,amssymb,amsopn,mathabx}
\usepackage{eucal,mathrsfs,makeidx,enumerate}

\usepackage{fullpage,color}
\usepackage[pdfstartview=FitH,
%             pdfauthor={\myauthor},
%             pdftitle={\mytitle},
            colorlinks,
            linkcolor=reference,
            citecolor=citation,
            urlcolor=e-mail,
            backref]{hyperref}
\usepackage[all]{xy}

\definecolor{todo}{rgb}{.80,.20,.20}
\definecolor{e-mail}{rgb}{0,.40,.80}
\definecolor{reference}{rgb}{.10,.40,.42}
\definecolor{mrnumber}{rgb}{.80,.40,0}
\definecolor{citation}{rgb}{0,.40,.80}

%%%%%%%%%%%%%%%%%%%%%%%%%%%%%%%%%%%%%%%%%%%%%%%%%%%%%%%%%%%%%%%%%%%%
% theorem stuff
%%%%%%%%%%%%%%%%%%%%%%%%%%%%%%%%%%%%%%%%%%%%%%%%%%%%%%%%%%%%%%%%%%%%

\theoremstyle{plain}

\newtheorem{thm}{Theorem}[section]
\newtheorem{defn}[thm]{Definition}
\newtheorem{deflem}[thm]{Definition/Lemma}
\newtheorem{notn}[thm]{Notation}
\newtheorem{lem}[thm]{Lemma}
\newtheorem{aside}[thm]{Aside}
\newtheorem{rem}[thm]{Remark}
\newtheorem{ex}[thm]{Example}
\newtheorem{facts}[thm]{Facts}
\newtheorem{cor}[thm]{Corollary}
\newtheorem{conj}[thm]{Conjecture}
\newtheorem{prop}[thm]{Proposition}


%%%%%%%%%%%%%%%%%%%%%%%%%%%%%%%%%%%%%%%%%%%%%%%%%%%%%%%%%%%%%%%%%%%%
% typography stuff
%%%%%%%%%%%%%%%%%%%%%%%%%%%%%%%%%%%%%%%%%%%%%%%%%%%%%%%%%%%%%%%%%%%%

\newcommand{\mb}[1]{\mathbf #1}
\newcommand{\mbb}[1]{\mathbb #1}
\newcommand{\mf}[1]{\mathfrak #1}
\newcommand{\mc}[1]{\mathcal #1}
\newcommand{\ms}[1]{\mathscr #1}
\newcommand{\mcu}[1]{\mathcu #1}
\newcommand{\oper}[1]{\operatorname{#1}}

\newcommand{\da}{\downarrow}
\newcommand{\ra}{\rightarrow}
\newcommand{\hra}{\hookrightarrow}
\newcommand{\dra}{\dashrightarrow}
\newcommand{\la}{\leftarrow}
\newcommand{\lra}{\longrightarrow}

\newcommand{\ov}{\overline}
\newcommand{\til}{\widetilde}
\newcommand{\wh}{\widehat}

\newcommand{\ZZ}{\mathbb{Z}}

\newcommand{\ann}{\oper{ann}}
\newcommand{\coker}{\oper{coker}}
\newcommand{\End}{\oper{End}}
\newcommand{\Aut}{\oper{Aut}}
\newcommand{\Stab}{\oper{Stab}}

\newcommand{\ind}{\oper{ind}}
\newcommand{\per}{\oper{per}}
\newcommand{\cores}{\oper{cor}}

\newcommand{\Br}{\oper{Br}}
\newcommand{\quat}[3]{
  \left(\begin{matrix} #1, #2 \\ #3
  \end{matrix}\right)
}
\newcommand{\symb}[3]{
  \left(#1, #2\right)_{#3}
}

\newcommand{\lcm}{\oper{lcm}}

%%%%%%%%%%%%%%%%%%%%%%%%%%%%%%%%%%%%%%%%%%%%%%%%%%%%%%%%%%%%%%%%%%%%
% other stuff
%%%%%%%%%%%%%%%%%%%%%%%%%%%%%%%%%%%%%%%%%%%%%%%%%%%%%%%%%%%%%%%%%%%%

\makeindex
\newcommand{\X}[1]{#1\index{#1}}
\newcommand{\Xb}[1]{\textbf{#1}\index{#1}}

\newcommand{\todo}[1]{\textcolor{todo}{#1}}

%%%%%%%%%%%%%%%%%%%%%%%%%%%%%%%%%%%%%%%%%%%%%%%%%%%%%%%%%%%%%%%%%%%%
% end preamble 
%%%%%%%%%%%%%%%%%%%%%%%%%%%%%%%%%%%%%%%%%%%%%%%%%%%%%%%%%%%%%%%%%%%%

\begin{document}

%%%%%%%%%%%%%%%%%%%%%%%%%%%%%%%%%%%%%%%%%%%%%%%%%%%%%%%%%%%%%%%%%%%%
% title stuff
%%%%%%%%%%%%%%%%%%%%%%%%%%%%%%%%%%%%%%%%%%%%%%%%%%%%%%%%%%%%%%%%%%%%


\author{Daniel Krashen}
\title{Graph Theory}

\maketitle
\tableofcontents

%%%%%%%%%%%%%%%%%%%%%%%%%%%%%%%%%%%%%%%%%%%%%%%%%%%%%%%%%%%%%%%%%%%%
% document stuff
%%%%%%%%%%%%%%%%%%%%%%%%%%%%%%%%%%%%%%%%%%%%%%%%%%%%%%%%%%%%%%%%%%%%


\chapter{Basic Notions}

\section{Preliminary notions and notations}

The substructure of the majority of modern mathematics is set theory. It
therefore would behoove us to take a very slight digression into some
useful concepts and notations.

\begin{defn}
A \Xb{set} is a collection of elements, is defined exactly by its elements.
Two sets are equal if they contain the same elements.
\end{defn}

\begin{defn}
For a set $S$, its \Xb{power set} $\ms P(S)$, is the set whose
elements are the subsets of $S$
\end{defn}

\begin{defn}
For a set $S$, we let $\ms P_k(S)$ denote its subsets with exactly $k$
elements.
\end{defn}

\begin{defn}
For sets $S, T$ we let $S \times T$ denote the set whose elements are
ordered pairs $(s, t)$ where $s \in S$ and $t \in T$.
\end{defn}

\begin{defn}
A \Xb{multiset} $\ms S$ is a pair $(S, m)$, where $S$ is a set, and $m$ is a
function $m: S \to \mathbb Z_{> 0}$ from $S$ to the positive integers. For
$s \in S$, we refer to $m(s)$ as the multiplicity of $s$ in $\ms S$. We
write $s \in \ms S$. We call $S$ the underlying set of $\ms S$.

For a multiset $\ms S = (S, m)$, we will use the notation $m_{\ms S}$ to
refer to $m$.
\end{defn}

\begin{defn}
Let $\ms S$ and $\ms T$ be multisets. We say that $\ms S \subset \ms T$ if
the underlying set of $\ms S$ is contained in the underlying set of $\ms
T$.
\end{defn}

If $S$ is a set, we will also identify $S$ with the multiset $(S, m)$
defined by $m(s) = 1$ for each $s \in S$ (that is, $S$ contains each of its
elements exactly $1$ time).

These mutisets are occasionally useful in combinatorics to think of the
idea of sampling with replacement/repetition.

\begin{defn}
If $\ms S = (S, m)$ is a multiset, we define the \Xb{cardinality} of $\ms
S$, denoted  $\#\ms S$, to be
\[ \sum_{s \in S} m(s). \]
\end{defn}
In particular, considering a set $T$ as a multiset as described above, we
have $\# T$ is exactly the number of elements of $T$.

\begin{defn} \label{multisubsets definition}
Let $S$ be a set. We write $\ms R(S)$ to denote the set of all mulisubsets
of $S$, and $\ms R_k(S)$ the set of all multisubsets of $S$ with
cardinality exactly $S$.
\end{defn}


\section{Definitions of graphs and examples}
Graphs encode the idea of connections between things, for example
\begin{itemize}
\item networks of computers
\item people and their relationships
\item cities and highways
\item sets and intersections
\item workers and tasks
\end{itemize}


In formal mathematical terms, a graph is:
\begin{defn}
A \Xb{graph} $G$ is an ordered triple $(V, E, \psi)$ consisting of 
\begin{itemize}
\item a set $V$, whose elements are referred to as vertices, 
\item a set $E$, whose elements are referred to as edges, and
\item an ``incidence'' function $\psi: E \to \ms R_2(V)$,
\end{itemize}
where $\ms R_2(V)$ is the set of unordered pairs of elements of $V$ (which
one may also think of as two elements multisubsets of $V$ -- see
Definition~\ref{multisubsets definition}).
\end{defn}

PICTURES AND EXAMPLES HERE

\begin{notn}
For a graph $G = (V, E, \psi)$ we write $V_G$ for $V$, $E_G$ for $E$ and
$\psi_G$ for $\psi$.
\end{notn}
In other words, using this notational convention, if we are given graphs
$G, H, K$, and have not specified letters for their sets of vertices,
edges, etcetera, we may write, for example, $E_K$ for the edges of the
graph $K$, $V_H$ for the vertices of $H$, and $\psi_G$ for the incidence
function of $G$.

\begin{defn}
Let $G$ be a graph, $e \in E_G$ an edge and $v \in V_G$ a vertex. We say
that $e$ and $v$ are \Xb{incident} if $v \in \psi(e)$.
\end{defn}

DIAGRAM

\begin{defn}
Let $G$ be a graph. If $e \in E_G$ is an edge, we say that $e$ is a
\Xb{loop}, if $e$ is incident to exactly one vertex.
\end{defn}

\begin{defn}
We say that $G$ is a \Xb{simple graph} if
\begin{itemize}
\item $G$ has no loops,
\item there is at most $1$ edge incident to any pair of vertices.
\end{itemize}
\end{defn}
Note that the second condition is the same as requiring that the function
$\psi_G$ be one-to-one.

Graphs can be drawn in many different ways:

\begin{defn}
$G$ is called a \Xb{planar graph} if it may be drawn in the plane with no
edges crossing.
\end{defn}

\section{Real world graph problems}

\subsection{Scheduling}
\begin{itemize}
\item vertices = jobs that need to be done
\item edges = jobs which require conflicting resources
\end{itemize}
problem: how to decide how many ``periods of work'' needed to complete
all jobs.

Similar problem: table arrangements at a wedding
\begin{itemize}
\item vertices = guests
\item edges = guest that don't get along
\end{itemize}
problem: how many tables?

translation: vertex colorings, chromatic number of a graph

\subsection{Tournaments}

various teams need to play each other. disjoint pairs of teams can play
simultaneously, but of course the same team can't play at the same time.
How many rounds are needed for teams to play each other?
\begin{itemize}
\item vertices = teams
\item edges = teams who need to play each other
\end{itemize}
problem: how many rounds?

\chapter{Subgraphs, new graphs from old}

\chapter{Proofs and formality}

\chapter{Spanning subgraphs and applications}

%\bibliographystyle{alpha}
%\bibliography{citations}
\printindex

\end{document}
